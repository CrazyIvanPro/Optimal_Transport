% Section 6: Conclusion

\section{Conclusion}
Optimal Transport Theory has gradually become a popular machine learning theoretical tool in recent years. 
In this project, we studied the theoretical background of Discrete Optimal Transport and implemented several methods for solving DOT problem.
To compare these algorithms, we conduct our experiment on artifical datasets and real images provided by DOTmark dataset.

Looking forward, we are aming at implement an algorithm which approaches scales in linear time. we are also expect to improve the stability and robustness 
of our algorithms with theoretical gurantees and technical feasibility. We have tried to implement the multiscale strategies method presented in \cite{article2} combined with existing methods, which try to
 establish a multi-layer transport problem so that the measure and cost between each two layers have
  a similar relationship, then the optimal solution of the original problem can be obtained after several 
  steps of "expand-correct" steps. Due to limited time, we couldn't present the implemenation here but the essence of this
  algorithm shows that the intrinsic feature of images in different coarsen level can be utilized for a faster convergence.

What's more, we want to utilize optimal transport as a powerful tool to conquer pratical problems in a bunch of applied areas.